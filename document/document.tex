\documentclass[a4paper,11pt]{article}
%\documentclass[a4paper,11pt]{book}

% macro for chinese
\usepackage{CJKutf8}
\begin{CJK}{UTF8}{gbsn}
%\begin{CJK}{UTF8}{song}
\CJKtilde

% macro for header/footer
\usepackage{fancyhdr}
% macro for C stype
\usepackage{listings}
%这条命令可以让LaTeX排版时将C++键字突出显示
\lstset{language=C}
%这条命令可以让LaTeX自动将长的代码行换行排版
\lstset{breaklines}
%这一条命令可以解决代码跨页时,章节标题,页眉等汉字不显示的问题
\lstset{extendedchars=false}

%使除首段以外的段落,首行缩进,
\usepackage{indentfirst}
%缩进两个汉字距离的格式。
\setlength{\parindent}{2em} 

% define the title
\author{arm小组}
\title{S3C2440裸机模板文档}

\begin{document}
% generates the title
\maketitle
% header/footer
\pagestyle{fancy}
\lhead{学生电子爱好者协会}

\newpage

% insert the table of contents
%\tableofcontents
% 第一章
\section{模板简介}
工程模板下有三个目录,boot,include,source。
\par
boot目录下的start.S是S3C2440的启动代码,boot.lds是链接文件,lowlevel\_init.s是寄存器配置文件;include目录下是工程的全部头文件;source目录下是程序源代码。
\par
顶层目录和各级子目录中的Makefile是编译管理文件。

% 1
\subsection{编译模板}
编译工程方法是在工程根目录下输入命令:
\begin{lstlisting}
	make all
\end{lstlisting}
\par
根目录下会生成boot.bin和boot\_efl两个可执行文件,boot.bin即为用于烧写的裸机程序。
\par
同时还会生成boot.map和一些.o文件。boot.map保存着裸机程序的连接地址,调试程序会用到。.o文件是编译过程中生成的临时文件。
\par
编译后的工程目录会生成各种文件,目录会显得比较混乱;调整工程结构后,生成的临时文件还会影响工程的编译,所以有时需要清理工程目录,删除编译工程产生的所有输出文件。删除所有生成文件的命令为:
\begin{lstlisting}
	make clean
\end{lstlisting}
\par
特别需要注意的是,该模板编译的可执行文件只能在内存中或者NOR Flash直接运行,不能烧写到NAND Flash上面运行(没有自启动功能)。
\par

% 2
\subsection{下载文件}
可执行文件的运行地址固定为0x37000000,也就是说必须下载到内存地址0x37000000才能正确运行。
下载方式具体查看开发板说明书,这里只是以GT2440开发板作示范。
\par
上位机使用ckermit作为终端软件,开发板搭载uboot。通过串口下载程序。
uboot启动后在ckermit上显示命令界面,输入下载命令,下载地址是0x37000000:
\begin{lstlisting}
	loadb 0x37000000
\end{lstlisting}
看到开发板等待下载的提示:
\begin{lstlisting}
	## Ready for binary (kermit) download to 0x37000000 at 115200 bps...
\end{lstlisting}
在ckermit中按组合键CTRL+$\backslash$ C退出连接进入ckermit命令界面。输入:
\begin{lstlisting}
	send /path/of/yours/boot.bin
\end{lstlisting}
其中/path/of/yours/boot.bin是烧写文件的路径。这个命令传输boot.bin文件到开发板内存0x37000000处。传输完成后,在ckermit命令界面下输入c重新连接上开发板,看到下载成功的提示:
\begin{lstlisting}
	## Total Size      = 0x00000374 = 884 Bytes
	## Start Addr      = 0x37000000
\end{lstlisting}
输入go命令跳转到内存0x37000000处执行:
\begin{lstlisting}
	go 0x37000000
\end{lstlisting}
从程序的执行效果可以看到,模板是一个流水灯程序。
\par
模板编译的所有程序的下载方式都是一样的,具有相同的文件名和相同的运行地址。同样的程序可以烧写进NOR Flash里,然后从Flash中启动。由于NOR Flash中往往装有~bootloader~,烧写NOR Flash会覆盖掉~bootloader~。所以为了使用方便,不建议调试裸机程序的时候把裸机程序烧写进NOR Flash。

%第二章
\section{编程接口}
arm9的处理器启动部分的代码千头万绪,gcc的工程管理同样晦涩难明,如果在入门阶段就去钻研启动代码和Makefile,就会陷入纷乱繁复的代码工程中无法自拔,不利于全身心的投入arm体系编程基础的学习。
\par
编译模板事先编写了启动部分的汇编代码和Makefile,通过提供有限的几个编程接口,使建立裸机程序的过程变得相当简单。
\par
启动代码,链接文件和Makfile等比较复杂的内容留到以后学习~bootloader~和内核移植的时候再来接触。

% 1
\subsection{增加源文件}
程序头文件目录是include,所有头文件都必须放在该目录下。
\par
程序源代码目录是source,所有源文件都必须建立在该目录下。源文件可以是C语言,也可以是汇编语言。
\par
将源代码加入工程需要在编译管理文件Makefile中增加编译条件。编译条件格式为单独一行:
\begin{lstlisting}
	obj += filename.o
\end{lstlisting}
其中filename是源文件名去掉.c或者.s的后缀。例如要将自己写的abc.c文件加入工程,就在Makefile中加入obj += abc.o的一行新条件;当在顶层目录执行make all命令后,abc.c文件就会被编译进boot.bin文件。
\par
如果要在source目录下建立多级目录,需要在新目录下建立相应的Makefile文件。子目录的Makefile编写方法和之前的Makefile一致,同时还要在上一层目录中的Makefile中包含子目录中的Makefile。
\par
多级子目录举例:在source目录下建立abc目录,目录下有源文件def.c。在abc目录下建立Makefile文件,内容为:
\begin{lstlisting}
	obj += def.o
\end{lstlisting}
同时还要在source目录下的Makefile中包含abc目录下的Makefile:
\begin{lstlisting}
	include source/abc/Makefile
\end{lstlisting}
这样在顶层目录运行make all之后,def.c的内容就会被编译进boot.bin。

% 2
\subsection{C编程}
在模板中,C语言程序的入口是main函数,标准声明格式为:
\begin{lstlisting}
	int main(void);
\end{lstlisting}
每个工程中只能有一个C语言入口函数,包含入口函数的C文件即为C代码入口文件。
\par
如果需要修改C程序入口函数格式,可以修改start.S文件。入口函数在汇编中的声明为:
\begin{lstlisting}
	_start_armboot:	.word main
\end{lstlisting}
将其中的main替换为相应的函数名即可改变C语言入口。
\par
S3C2440的寄存器地址和中断地址声明在include/2440addr.h中,其调用方式类似于MSP430的库文件调用。

% 3
\subsection{汇编编程}
在模板中,汇编语言的格式是arm-linux汇编,与《arm体系结构与编程》的内容有着细微的差别。
\par
汇编函数的声明为:
\begin{lstlisting}
	asm_function	.word _asm_function
	.globl		asm_function
	_asm_function:
		function_body
		mov	pc, lr
\end{lstlisting}
声明的汇编函数可以在C文件中调用,调用前需要在C文件中声明:
\begin{lstlisting}
	extern void asm_function(void);
\end{lstlisting}
\par
汇编文件中声明C函数的方式为
\begin{lstlisting}
	_c_function	.word c_function
\end{lstlisting}
在汇编文件中声明后的C函数能够在汇编文件中调用。C函数和汇编函数的参数传递遵循ATPCS原则。
\par
C及汇编的混合编程参考《arm体系结构与编程》的第八章。


\clearpage
\end{CJK}
\end{document}
